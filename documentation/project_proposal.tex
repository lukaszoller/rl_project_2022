% !TEX TS-program = pdflatex
% !TEX encoding = UTF-8 Unicode

% This is a simple template for a LaTeX document using the "article" class.
% See "book", "report", "letter" for other types of document.

\documentclass[11pt]{article} % use larger type; default would be 10pt

\usepackage[utf8]{inputenc} % set input encoding (not needed with XeLaTeX)

%%% Examples of Article customizations
% These packages are optional, depending whether you want the features they provide.
% See the LaTeX Companion or other references for full information.

%%% PAGE DIMENSIONS
\usepackage{geometry} % to change the page dimensions
\geometry{a4paper} % or letterpaper (US) or a5paper or....
% \geometry{margin=2in} % for example, change the margins to 2 inches all round
% \geometry{landscape} % set up the page for landscape
%   read geometry.pdf for detailed page layout information

\usepackage{graphicx} % support the \includegraphics command and options

% \usepackage[parfill]{parskip} % Activate to begin paragraphs with an empty line rather than an indent

%%% PACKAGES
\usepackage{booktabs} % for much better looking tables
\usepackage{array} % for better arrays (eg matrices) in maths
\usepackage{paralist} % very flexible & customisable lists (eg. enumerate/itemize, etc.)
\usepackage{verbatim} % adds environment for commenting out blocks of text & for better verbatim
\usepackage{subfig} % make it possible to include more than one captioned figure/table in a single float
% These packages are all incorporated in the memoir class to one degree or another...

%%% HEADERS & FOOTERS
\usepackage{fancyhdr} % This should be set AFTER setting up the page geometry
\pagestyle{fancy} % options: empty , plain , fancy
\renewcommand{\headrulewidth}{0pt} % customise the layout...
\lhead{}\chead{}\rhead{}
\lfoot{}\cfoot{\thepage}\rfoot{}

%%% SECTION TITLE APPEARANCE
\usepackage{sectsty}
\allsectionsfont{\sffamily\mdseries\upshape} % (See the fntguide.pdf for font help)
% (This matches ConTeXt defaults)

%%% ToC (table of contents) APPEARANCE
\usepackage[nottoc,notlof,notlot]{tocbibind} % Put the bibliography in the ToC
\usepackage[titles,subfigure]{tocloft} % Alter the style of the Table of Contents
\renewcommand{\cftsecfont}{\rmfamily\mdseries\upshape}
\renewcommand{\cftsecpagefont}{\rmfamily\mdseries\upshape} % No bold!


\usepackage{graphicx}         % integration of images
\usepackage{float}			% place pictures at specific place in text
\usepackage{amsmath}	% mathematical equations
\usepackage{xcolor} 	% colored font


%%% END Article customizations

%%% The "real" document content comes below...

\title{Project Proposal}
\author{Jakub Tłuczek, Lukas Zoller}
%\date{} % Activate to display a given date or no date (if empty),
         % otherwise the current date is printed 

\begin{document}
\maketitle

\section{Project Goal}
The goal of our project is to optimize the parameterization of the implemented Reinforcement Learning algorithm. For that we will run experiments with different parameterization of update function and environment where an agent will act in an environment (both described below). The experiments will be visualized and this visualizations will be used to find the optimal parameters of the update function.

\section{Environment}
Out environment represents a quadratic sea map where a pirate (the agent) wants to rob and destroy merchand ships. Each enemy ship will return a positive reward. If the agent robs all merchand ships, the game will be over. There are also enemy ships moving on the map which the agent has to avoid. If the agent is on the same field as an enemy ship the game will also be over with a negative reward for the agent.

The map size as well as the number of enemy and merchand ships shall be random (in some to be defined limits). Also the environment is stochastic in a sense that the agents actions (up, down, left, right) will only be executed with a certain probability p (i.e. 0.7). The other actions will be executed with a probability $\frac{1-p}{n-1}$, where n is the number of actions. 

The agent does not see the whole map but only a certain area around himself (square). \todo{We did not discuss this. I would also be ok for me to see the whole map.}

\section{Algorithm}
We want to use Soft Actor Critic (SAC) as Reinforcement Learning algorithm because of the following reasons: First SAC is a model-free algorithm which is well suited for an environment like ours which changes over time. Second, also because of the changing environment we want to use an online algorithm which allows us to update the policy every time step. Third SAC incorporates some new interesting ideas like the maximization of policy entropy to enable exploration.


\end{document}
